\chapter{Experimental features}
\label{chap:expfeatures}

Experimental features are currently under active development. They are expected to change rapidly, and they have no full support.

\section{Hierarchical learning algorithms}

A hierarchical learning algorithm (HLA) can be any tree structure of models, where an input is propagated through the tree until a final decision is taken on a leaf node. At every node, a learning algorithm is employed to choose which branch to follow. To understand its utility, try to plot the output values of the \textit{uci\_yacht} dataset: you will see that there is a strong gap between positive and negative values. In this case, it may be better to envision a two-stage architecture:

\begin{itemize}
\item The first level discriminates between positive and negative output values.
\item The second level has two learning algorithms trained only on positive and negative outputs respectively.
\end{itemize}

\noindent Currently, you can implement this with the following code:

\begin{lstlisting}
add_model('NODE', 'Internal node', @ExtremeLearningMachine);
add_model('LEAF1', 'First leaf', @ExtremeLearningMachine);
add_model('LEAF2', 'Second leaf', @ExtremeLearningMachine);

construct_hierarchical_algorithm([2 0 0], {RangeAggregator(0)}, {'NODE', 'LEAF1', 'LEAF2'});

add_dataset('Y', 'Yacht (UCI)', 'uci_yacht');
\end{lstlisting}

\noindent The parameters to the \verb|construct_hierarchical_algorithm| method are:

\begin{itemize}
\item A cell array describing the number of childs of the tree, in a depth-first fashion.
\item A cell array of \verb|Aggregator| objects. Aggregators are used to associate to every original output a particular branch of the node. In our case, \verb|RangeAggregator| discriminates between negative and positive values, sending them to the left and right branches of the node.
\item A cell array containing the ids of the models composing the tree.
\end{itemize}

\noindent We can also add an output script for plotting the tree structure:

\begin{lstlisting}
add_feature(ExecuteOutputScripts('info_hla'));
\end{lstlisting}